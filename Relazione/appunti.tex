\documentclass[a4paper,12pt]{article}
\usepackage[utf8]{inputenc}
\usepackage[T1]{fontenc}
\usepackage[english]{babel}

% Questo comando importa i miei comandi personali che semplificano due o tre cagate
\usepackage{float}
\usepackage{url}
\usepackage{xcolor}
\usepackage{pdfpages}
\usepackage{graphicx}
\usepackage{mathtools,amssymb}
\usepackage{amsmath}

%Pro and cons
\usepackage{amssymb}% http://ctan.org/pkg/amssymb
\usepackage{pifont}% http://ctan.org/pkg/pifont
\newcommand{\cmark}{\ding{51}}%
\newcommand{\xmark}{\ding{55}}%
\newcommand{\pro}{\item[\cmark]}
\newcommand{\con}{\item[\xmark]}

\renewcommand{\tilde}{$\sim$}

%sets
\newcommand{\set}[1]{\{#1\}}

% Per numerare i paragrafi
\setcounter{tocdepth}{4}
\setcounter{secnumdepth}{4}
\let\oldpar\paragraph
\renewcommand{\paragraph}[1]{\oldpar{#1}\mbox{}\\}

%Per metter una freccia in modo fast
\newcommand{\fdx}{$\rightarrow$\ }

%Comando per creare nuove definizioni stile blocco
\newcommand{\definition}[2]{
	\begin{table}[H]
	\centering
		\begin{tabular}{|p{0.9\linewidth}}
		\textbf{#1}\\ %Titolo della definzione
		#2\\%Testo della definizione
		\end{tabular}
	\end{table}
	\noindent
}

%Comando esempio
\newcommand{\esempio}[1]{
	\begin{table}[H]
	\centering
		\begin{tabular}{p{0.9\linewidth}}
		\textbf{EX}\\ %Titolo della definzione
		\hline
		#1\\%Testo della definizione
		\end{tabular}
	\end{table}
	\noindent
}

\newcommand{\df}[2]{\definition{#1}{#2}}

%%%%%%%%%%%%%%%%%%%%%%%%%%%%%%%%%%%%%%%%%%%%%%%%%%%%%%%%%%%%%%%%%%%%%
%      INSBOX --- macros for inserting pictures into paragraphs     %
%       Micha\l{} Gulczy\'nski, Szczecin, Jan 1996 / Feb 1998       %
%                     mgulcz@we.tuniv.szczecin.pl                   %
%%%%%%%%%%%%%%%%%%%%%%%%%%%%%%%%%%%%%%%%%%%%%%%%%%%%%%%%%%%%%%%%%%%%%
%
%  version 2.2
%
%  available macros:
%    * \InsertBoxC{anybox}
%        insert a centered box (use int _inside_ a paragraph)
%    * \InsertBoxL{after_line}{anybox}[correction]
%    * \InsertBoxR{after_line}{anybox}[correction]
%        insert a box in the left/right after specified number of lines;
%        correction specified in square brackets is optional;
%        both macros should be called _before_ a paragraph
%    * \MoveBelowBox
%        start a new paragraph just below the current frame
%
%  see the demo.tex file for more information
%

\catcode`\@ = 11
%
%  Margin between the text and the box:
\newdimen\@InsertBoxMargin
\@InsertBoxMargin = 2mm
%
%  definition of \ParShape, an inproved version of plain \parshape
%
\newcount\@numlines    % sum: m_1+...+m_n
\newcount\@linesleft   % counter used when reading lines of \ParShape
\def\ParShape{%
    \@numlines = 0
    \def\@parshapedata{ }% here we'll collect data for plain \parshape
    \afterassignment\@beginParShape
    \@linesleft
}%
\def\@beginParShape{%
    \ifnum \@linesleft = 0
      \let\@whatnext = \@endParShape
    \else
      \let\@whatnext = \@readnextline
    \fi
    \@whatnext
}%
\def\@endParShape{%
    \global\parshape = \@numlines \@parshapedata
}%
\def\@readnextline#1 #2 #3 {% #1 #2 #3 are: m_i, leftskip_i, rightskip_i
    \ifnum #1 > 0
      \bgroup  % I want to keep changes of \dimen0 and \count0 local
        \dimen0 = \hsize
        \advance \dimen0 by -#2  % \parshape requires left skip and
        \advance \dimen0 by -#3  % _length_of_line_ (not right skip!)
        \count0 = 0
        \loop
          \global\edef\@parshapedata{%
            \@parshapedata    % add to \@parshapedata:
            #2                % left skip
            \space            % a space
            \the\dimen0       % length of line
            \space            % another space
          }%
          \advance \count0 by 1
          \ifnum \count0 < #1
        \repeat
      \egroup
      \advance \@numlines by #1
    \fi
    \advance \@linesleft by -1
    \@beginParShape
}%
%
%  \InsertBoxC, \InsertBoxL, \InsertBoxR
%
\newbox\@boxcontent     % box containing the picture to be inserted
\newcount\@numnormal    % number of leading lines to typeset normally
\newdimen\@framewidth   % width of the frame
\newdimen\@wherebottom  % position of frame's bottom
\newif\if@byframe       % true if we are just beside the frame
\@byframefalse
%
%
\def\InsertBoxC#1{%
  \leavevmode
  \vadjust{
    \vskip \@InsertBoxMargin
    \hbox to \hsize{\hss#1\hss}
    \vskip \@InsertBoxMargin
  }%
}%
\def\InsertBoxL#1#2{%
  \@numnormal = #1
  \setbox\@boxcontent = \hbox{#2}%
  \let\@side = 0
  \futurelet \@optionalparameter \@InsertBox
}
\def\InsertBoxR#1#2{%
  \@numnormal = #1
  \setbox\@boxcontent = \hbox{#2}%
  \let\@side = 1
  \futurelet \@optionalparameter \@InsertBox
}%
\def\@InsertBox{%
  \ifx \@optionalparameter [
    \let\@whatnext = \@@InsertBoxCorrection
  \else
    \let\@whatnext = \@@InsertBoxNoCorrection
  \fi
  \@whatnext
}%
\def\@@InsertBoxCorrection[#1]{%
  \ifx \@side 0
    \@@InsertBox{#1}{0}{{\the\@framewidth} 0cm}%
  \else
    \@@InsertBox{#1}{1}{0cm {\the\@framewidth}}%
  \fi
}%
\def\@@InsertBoxNoCorrection{%
  \@@InsertBoxCorrection[0]%
}%
\def\@@InsertBox#1#2#3{%
  \MoveBelowBox
  \@byframetrue
  % \@wherebottom = \pagetotal + (\@numnormal * \baselineskip) +
  %                 (height of \@boxcontent) + (2 * \@InsertBoxMargin)
  \@wherebottom = \baselineskip
  \multiply \@wherebottom by \@numnormal
  \advance \@wherebottom by 2\@InsertBoxMargin
  \advance \@wherebottom by \ht\@boxcontent
  \advance \@wherebottom by \pagetotal
  % I have no idea why, but \InsertBox called at the top of a page
  % calculates space for the box one line too big
  \ifdim \pagetotal = 0cm
    \advance \@wherebottom by -\baselineskip  % ^ reduction
  \fi
  % add the correction
  \advance \@wherebottom by #1\baselineskip
  % \@framewidth = (width of \@boxcontent} + \@InsertboxMargin
  \@framewidth = \wd\@boxcontent
  \advance \@framewidth by \@InsertBoxMargin
  %
  \bgroup  % to keep changes of \dimen0 local
    % check if the box fits in the page
    \ifdim \pagetotal = 0cm
      \dimen0 = \vsize
    \else
      \dimen0 = \pagegoal
    \fi
    \ifdim \@wherebottom > \dimen0
      % print a warning message ...
      \immediate\write16{+--------------------------------------------------------------+}%
      \immediate\write16{| The box will not fit in the page. Please, re-edit your text. |}%
      \immediate\write16{+--------------------------------------------------------------+}%
      % ... and mark this place in document with a black box
      \vrule width \overfullrule
    \fi
  \egroup
  \prevgraf = 0
  % insert the box in the left (if #2 = 0) or in the right (if #2 = 1)
  \vbox to 0cm{%
    \dimen0 = \baselineskip
    \multiply \dimen0 by \@numnormal
    \advance \dimen0 by -\baselineskip
    \setbox0 = \hbox{y}%
    \vskip \dp0
    \vskip \dimen0
    \vskip \@InsertBoxMargin
    \ifnum #2 = 1
      \vtop{\noindent \hbox to \hsize{\hss \box\@boxcontent}}%
    \else
      \vtop{\noindent \box\@boxcontent}%
    \fi
    \vss
  }%
  % I have no idea why, but this is really necessary
  \vglue -\parskip
  \vskip -\baselineskip
  % each following paragraph needs to be formatted properly
  \everypar = {%
    % are we already below the bottom of the box?
    \ifdim \pagetotal < \@wherebottom
      % no...
      \bgroup  % to keep some changes local
        % let's calculate parameters for \ParShape
        \dimen0 = \@wherebottom
        \advance \dimen0 by -\pagetotal
        \divide \dimen0 by \baselineskip
        \count1 = \dimen0
        \advance \count1 by 1
        \advance \count1 by -\@numnormal
        \ifnum #2 = 1
          \ParShape = 3
                      {\the\@numnormal}   0cm   0cm
                      {\the\count1}       0cm   {\the\@framewidth}
                      1                   0cm   0cm
        \else
          \ParShape = 3
                      {\the\@numnormal}   0cm                  0cm
                      {\the\count1}       {\the\@framewidth}   0cm
                      1                   0cm                  0cm
        \fi
      \egroup
    \else
      % yes!
      \@restore@    % it's time to end everything
    \fi
  }%
  % this definition isn't very necessary --- just in case the paragraph
  % following \InsertBoxL or \InsertBoxR has fewer lines that the
  % first argument of the macro
  \def\par{%
      \endgraf
      \global\advance \@numnormal by -\prevgraf
      \ifnum \@numnormal < 0
        \global\@numnormal = 0
      \fi
      \prevgraf = 0
  }%
}%
%
%  call this macro to move the current position just below the
%  current frame
%
\def\MoveBelowBox{%
  \par
  \if@byframe
    \global\advance \@wherebottom by -\pagetotal
    \ifdim \@wherebottom > 0cm
      \vskip \@wherebottom
    \fi
    \@restore@
  \fi
}%
%
%  normal settings are as follows:
%
\def\@restore@{%
    \global\@wherebottom = 0cm
    \global\@byframefalse
    \global\everypar = {}%
    \global\let \par = \endgraf
    \global\parshape = 1 0cm \hsize
}%
%
%  someone told me that in LaTeX there is no \pageno counter;
%  the counterpart is \c@page
%
\ifx \documentclass \@Dont@Know@What@It@Is@
\else
  \let \pageno = \c@page
\fi


\catcode`\@ = 12

\newcommand{\lessonDate}[1]{\InsertBoxR{0}{\tiny{#1}}}

\newcommand{\E}{\`E\space}

\usepackage{listings}
\lstset{language=C++,
                keywordstyle=\color{blue},
                stringstyle=\color{red},
                commentstyle=\color{green},
                morecomment=[l][\color{magenta}]{\#}
}
\usepackage{hyperref}
\usepackage{cleveref}

% Tema scuro
%\usepackage{xcolor}
%
%\pagecolor[rgb]{0.25,0.25,0.25}
%\color[rgb]{1,1,1}



\sloppy
\begin{document}

\begin{titlepage}
\begin{center}
	\Large{\textbf{Documentation for Algorithms for Massive Datasets project: Turkish lira recognizer}}
\vfill
\normalsize{Caccaro Sebastiano}\\
\normalsize{Cavagnino Matteo}\\
\normalsize{A.A.2019/2020}
\end{center}
\end{titlepage}
\pagenumbering{Roman}

\vspace*{\fill}
\textit{We declare that this material, which We now submit for assessment, is entirely our own work and has not been taken from the work of others, save and to the extent that such work has been cited and acknowledged within the text of our work. We understand that plagiarism, collusion, and copying are grave and serious offences in the university and accept the penalties that would be imposed should we engage in plagiarism, collusion or copying. This assignment, or any part of it, has not been previously submitted by us or any other person for assessment on this or any other course of study.}
\vspace*{\fill}

\newpage

\tableofcontents

\clearpage
%magari c'è qualcosa di un po piu slick di una section per questo

\pagenumbering{arabic}

\newpage
\section{Introduction}
The Objective of this project is to build a Turkish Lira banknotes image recognizer through a Convolutional Neural Network.
The proposed solution, based on Tensorflow libraries, contains steps to dinamically download the dataset, preprocess the images in it and use the processed images to train a 
Convolutional Neural Network in recognizing and classifying them.
Since the given dataset contains many images and since the request is to classify some precise details of them, it's expected to achieve good results from the proposed solution and in particular
from the proposed model.

\newpage
\section{The Turkish Lira banknotes dataset}
The chosen dataset <add ref> is originally composed of 6000 images of Turkish Lira banknotes, 
organized in folders grouping banknotes by their value and already splitted in training and validation set.

\newpage
\subsection{Preprocessing techniques applied to the dataset}
To start the pre-processing phase, it's needed to scale the images to a more appropriate size to not overload the machine memory; the size scale factor used in this project is 5.
The given dataset was provided with two text files listing the images belonging to the training and to the validation datasets; using this lists, the images files have been divided in the
two respective labeled datasets. 
The training dataset has been created using a "prefetch dataset" and the validation dataset has been created using a "repeat dataset" both using also batching and caching techniques.


\newpage
\subsection{Considered algorithms and their implmentation}
?

%meglio fare una sezione a riguardo secondo me, ci mettiamo tutta la parte sul batching, memorizzazione , uso di disco e ram e scaling delle immagini  
\newpage
\section{Scalability of the proposed solution}
The Scalability of this project is granted by (batching, caching, img scaling, ? ...) 

%esporre i modelli usati e i relativi risultati in ordine temporale in modo da fornire una sequenza dei ragionamenti fatti
\newpage
\section{Experiments and results}
Different models have been tested during the developing process; in this section some of those will be shown and the relative results will be discussed.
(aggiungere modelli e screen dei grafici che fanno spessore e cosa va e cosa non va nei vari modelli)

%solita roba da conclusioni
\newpage
\section{Conclusions}
As seen in the previous sections, after a few experiments, this project has been succesful in classifying the images of Turkish 
Lira banknotes with high accuracy and low data loss.
(altro?)




Esempio di citazione\cite{latexcompanion}

\newpage
\addcontentsline{toc}{section}{References}
\begin{thebibliography}{9}
\bibitem{latexcompanion} 
Michel Goossens, Frank Mittelbach, and Alexander Samarin. 
\textit{The \LaTeX\ Companion}. 
Addison-Wesley, Reading, Massachusetts, 1993.

\bibitem{einstein} 
Albert Einstein. 
\textit{Zur Elektrodynamik bewegter K{\"o}rper}. (German) 
[\textit{On the electrodynamics of moving bodies}]. 
Annalen der Physik, 322(10):891–921, 1905.

\bibitem{knuthwebsite} 
Knuth: Computers and Typesetting,
\\\texttt{http://www-cs-faculty.stanford.edu/\~{}uno/abcde.html}
\end{thebibliography}


\end{document}
