\documentclass[a4paper,12pt]{article}
\usepackage[utf8]{inputenc}
\usepackage[T1]{fontenc}
\usepackage[english]{babel}

% Questo comando importa i miei comandi personali che semplificano due o tre cagate
\usepackage{float}
\usepackage{url}
\usepackage{xcolor}
\usepackage{pdfpages}
\usepackage{graphicx}
\usepackage{mathtools,amssymb}
\usepackage{amsmath}

%Pro and cons
\usepackage{amssymb}% http://ctan.org/pkg/amssymb
\usepackage{pifont}% http://ctan.org/pkg/pifont
\newcommand{\cmark}{\ding{51}}%
\newcommand{\xmark}{\ding{55}}%
\newcommand{\pro}{\item[\cmark]}
\newcommand{\con}{\item[\xmark]}

\renewcommand{\tilde}{$\sim$}

%sets
\newcommand{\set}[1]{\{#1\}}

% Per numerare i paragrafi
\setcounter{tocdepth}{4}
\setcounter{secnumdepth}{4}
\let\oldpar\paragraph
\renewcommand{\paragraph}[1]{\oldpar{#1}\mbox{}\\}

%Per metter una freccia in modo fast
\newcommand{\fdx}{$\rightarrow$\ }

%Comando per creare nuove definizioni stile blocco
\newcommand{\definition}[2]{
	\begin{table}[H]
	\centering
		\begin{tabular}{|p{0.9\linewidth}}
		\textbf{#1}\\ %Titolo della definzione
		#2\\%Testo della definizione
		\end{tabular}
	\end{table}
	\noindent
}

%Comando esempio
\newcommand{\esempio}[1]{
	\begin{table}[H]
	\centering
		\begin{tabular}{p{0.9\linewidth}}
		\textbf{EX}\\ %Titolo della definzione
		\hline
		#1\\%Testo della definizione
		\end{tabular}
	\end{table}
	\noindent
}

\newcommand{\df}[2]{\definition{#1}{#2}}

\input{insbox.tex}
\newcommand{\lessonDate}[1]{\InsertBoxR{0}{\tiny{#1}}}

\newcommand{\E}{\`E\space}

\usepackage{listings}
\lstset{language=C++,
                keywordstyle=\color{blue},
                stringstyle=\color{red},
                commentstyle=\color{green},
                morecomment=[l][\color{magenta}]{\#}
}
\usepackage{hyperref}
\usepackage{cleveref}

% Tema scuro
%\usepackage{xcolor}
%
%\pagecolor[rgb]{0.25,0.25,0.25}
%\color[rgb]{1,1,1}



\sloppy
\begin{document}

\begin{titlepage}
\begin{center}
	\Large{\textbf{Documentation for Algorithms for Massive Datasets project: Turkish lira recognizer}}
\vfill
\normalsize{Caccaro Sebastiano}\\
\normalsize{Cavagnino Matteo}\\
\normalsize{A.A.2019/2020}
\end{center}
\end{titlepage}
\pagenumbering{Roman}

\vspace*{\fill}
\textit{We declare that this material, which We now submit for assessment, is entirely our own work and has not been taken from the work of others, save and to the extent that such work has been cited and acknowledged within the text of our work. We understand that plagiarism, collusion, and copying are grave and serious offences in the university and accept the penalties that would be imposed should we engage in plagiarism, collusion or copying. This assignment, or any part of it, has not been previously submitted by us or any other person for assessment on this or any other course of study.}
\vspace*{\fill}

\newpage

\tableofcontents

\clearpage
%magari c'è qualcosa di un po piu slick di una section per questo

\pagenumbering{arabic}

\newpage
\section{Introduction}
The Objective of this project is to build a Turkish Lira banknotes image recognizer through a Convolutional Neural Network.
The proposed solution, based on Tensorflow libraries, contains steps to dinamically download the dataset, preprocess the images in it and use the processed images to train a 
Convolutional Neural Network in recognizing and classifying them.
Since the given dataset contains many images and since the request is to classify some precise details of them, it's expected to achieve good results from the proposed solution and in particular
from the proposed model.

\newpage
\section{The Turkish Lira banknotes dataset}
The chosen dataset <add ref> is originally composed of 6000 images of Turkish Lira banknotes, 
organized in folders grouping banknotes by their value and already splitted in training and validation set.

\newpage
\subsection{Preprocessing techniques applied to the dataset}
In this section the following preprocessing techniques applied to the considered dataset will be discussed:
\begin{itemize}
	\item image scaling
	\item training and validation sets creation
\end{itemize}

Regarding the image scaling, the given images have a size of 720x1280; to not overload the memory of the computing machine it has been reduced by 5 times resulting in a size of 140x256 per image.
For the training and validation sets, they have been created starting from two text files, provided with the dataset, listing all the images that needed to be used for the training or validation phases.
The training dataset array has then been processed using shuffling, batching and repeating techniques to prepare it for the learning process.
The validation dataset array has instead been processed using only batching and repeating; for both the datasets, the batches size has been 
set to 32 images per batch.

The training dataset has also been processed using the prefetch technique, this
allows later elements to be prepared while the current element is being
processed. This often improves latency and throughput, at the cost of
using additional memory to store prefetched elements.



\newpage
\subsection{Considered algorithms and their implmentation}
?

%meglio fare una sezione a riguardo secondo me, ci mettiamo tutta la parte sul batching, memorizzazione , uso di disco e ram e scaling delle immagini  
\newpage
\section{Scalability of the proposed solution}
The Scalability of this project is granted by (batching, caching, img scaling, ? ...) 

%esporre i modelli usati e i relativi risultati in ordine temporale in modo da fornire una sequenza dei ragionamenti fatti
\newpage
\section{Experiments and results}
Different models have been tested during the developing process; in this section some of those will be shown and the relative results will be discussed.
\subsection{Model summary}
For each tested model the following data will be reported:
\begin{itemize}
\item The NN architecture
\item Hyperparameters used
\item Data on accuracy for three repeated runs
\item Graph of one of the runs
\item Comment on the architecture and results
\end{itemize}
Each model will be trained for exactly 20 epochs in order to get consistent results.\\
In the layer tables the input  layer will not be reported, as it always corresponds to resized image size (\texttt{144,256,3}).\\
Also note that some abbreviations are used in the Layer Config field in order to for the table to fit:
\begin{itemize}
\item \texttt{k} stands for \texttt{kernel size}
\item \texttt{s} stands for \texttt{strides}
\item \texttt{f} stands for \texttt{filters}
\item \texttt{p} stands for \texttt{pool size}
\end{itemize}

%Commands needed for table
\newcommand{\conv}{Convolution(\texttt{Conv2d})}
\newcommand{\convP}[3]{\texttt{k=#1, s=#2, f=#3}}
\newcommand{\convKSF}[3]{\convP{#1}{#2}{#3}}

\newcommand{\flt}{Flatten(\texttt{Flatten})}

\newcommand{\dns}{Dense(\texttt{Dense})}
\newcommand{\dnsP}[1]{\texttt{u=#1}}

\newcommand{\pool}{MaxPooling(\texttt{MaxPooling2D})}
\newcommand{\poolN}{\texttt{p=2x2}}

\subsection{Models}

\subsubsection{Baseline Model}
\begin{table}[H]
    \centering
	\begin{tabular}{lcccc}
	\textbf{Layer Type} & \textbf{Layer Config} & \textbf{Activation}  & \textbf{Output} & \textbf{Params}\\ \hline
	\conv	& \convP{5}{3}{5}	& relu		& \texttt{48,86,5} 	& \texttt{380}\\
	\flt		& /					& relu		& \texttt{20640}		& \texttt{0}\\
	\dns		& \dnsP{64}			& relu		& \texttt{64}		& \texttt{1321024}\\
	\dns		& \dnsP{6}			& softmax	& \texttt{64}		& \texttt{390}\\
	\end{tabular}
	%Total params: 1,321,794
	%Trainable params: 1,321,794
	%Non-trainable params: 0
\end{table}

\begin{table}[H]
	\centering
	\begin{tabular}{lc}
	\textbf{Param} & \textbf{Value}\\ \hline
	Batch Size 	& 32 \\
	Optimizer 	& Adam \\
	Base lr		& 0.001 \\
	\end{tabular}
\end{table}


\begin{figure}[H]
	\begin{center}
	\includegraphics[width=\linewidth]{Immagini/Baseline-1}
	\caption{Graph of the first run}
	\end{center}
\end{figure}
\begin{table}[H]
	\centering
	\begin{tabular}{cccccc}
		\textbf{Run} &\textbf{Loss}&\textbf{V.Loss} &\textbf{Acc.}&\textbf{V.Acc.}&\textbf{$\Delta$ Acc.} \\ \hline
		1	& 0.0016		& 0.0530		& 1.0000		& 0.9754		& 0.0246 \\
		2	& 0.0012		& 0.0342		& 1.0000		& 0.9821		& 0.0179 \\
		3	& 0.0060		& 0.0497		& 0.9996		& 0.9866		& 0.0130 \\
		\textbf{Avg} & \textbf{0.0029}	& \textbf{0.0456}	& \textbf{0.9996} 	& \textbf{0.9814}	& \textbf{0.0185} 
	\end{tabular}
\end{table}




\subsubsection{Convolution Model}
\begin{table}[H]
    \centering
	\begin{tabular}{lcccc}
	\textbf{Layer Type} & \textbf{Layer Config} & \textbf{Activation}  & \textbf{Output} & \textbf{Params}\\ \hline
	\conv	& \convKSF{5}{3}{5}	& relu		& \texttt{48,86,5} 	& \texttt{380}\\
	\conv	& \convKSF{5}{2}{8}	& relu		& \texttt{24,43,8} 	& \texttt{1008}\\	
	\conv	& \convKSF{3}{1}{12}	& relu		& \texttt{24,43,12} 	& \texttt{876}\\
	\conv	& \convKSF{3}{1}{15}	& relu		& \texttt{24,43,15} 	& \texttt{1635}\\
	\conv	& \convKSF{3}{1}{18}	& relu		& \texttt{24,43,18} 	& \texttt{2448}\\
	
	\flt		& /					& /		& \texttt{20640}		& \texttt{0}\\
	\dns		& \dnsP{64}			& relu		& \texttt{64}		& \texttt{1188928}\\
	\dns		& \dnsP{6}			& softmax	& \texttt{6}		& \texttt{390}\\
	\multicolumn{4}{r}{\textbf{TOTAL}}&{\textbf{1,195,665}}\\
	\end{tabular}
	%Total params: 1,195,665
	%Trainable params: 1,195,665
	%Non-trainable params: 0
\end{table}


\begin{table}[H]
	\centering
	\begin{tabular}{lc}
	\textbf{Param} & \textbf{Value}\\ \hline
	Batch Size 	& 32 \\
	Optimizer 	& Adam \\
	Base lr		& 0.001 \\
	Epochs		& 20 \\
	\end{tabular}
\end{table}


\begin{figure}[H]
	\begin{center}
	\includegraphics[width=\linewidth]{Immagini/conv-3}
	\caption{Graph of the third run}
	\end{center}
\end{figure}
\begin{table}[H]
	\centering
	\begin{tabular}{cccccc}
		\textbf{Run} &\textbf{Loss}&\textbf{V.Loss} &\textbf{Acc.}&\textbf{V.Acc.}&\textbf{$\Delta$ Acc.} \\ \hline
		1   & 1.0908e-04    &   0.0439  & 1.0000    & 0.9866    & 0.0134 \\
		2   & 8.7761e-05    &   0.0912  & 1.0000    & 0.9888    & 0.0112 \\
		3   & 1.8083e-04    &   0.0609  & 1.0000    & 0.9911    & 0.0089 \\
		\textbf{Avg} & \textbf{1.2589e-04} & \textbf{0.0653}	& \textbf{1.0000}	& \textbf{0.9889} 	& \textbf{0.0112} 
	\end{tabular}
\end{table}

This models is an iteration of the baseline model, achieved by adding four more convolution. The convolution have been added with the following logic in mind:
\begin{itemize}
\item The first two convolutions have a stride greater than 1, and that leads to a minimal reductions in the size of the tensor. This in turn should limit the number of parameters needed for training and in turn reduce overfitting.
\item The layers are set in way to progressively increase the number of channel and reduce height and width.
\end{itemize}
The results show the model is slightly less overfitting than the baseline. Both models reached 1.0000 accuracy on the training set, so this translate in an increased validation accuracy.
%NOTARE CHE AGGIUNGERE PIU LIVELLI NON AIUTA





\subsubsection{Convolution and Pooling Model}
\begin{table}[H]
    \centering
	\begin{tabular}{lcccc}
	\textbf{Layer Type} & \textbf{Layer Config} & \textbf{Activation}  & \textbf{Output} & \textbf{Params}\\ \hline
	\conv	& \convKSF{5}{1}{5}	& relu		& \texttt{144,256,5} 	& \texttt{380}\\
	\pool	& \poolN				&	/		& \texttt{72,128,8}		& 0	\\
	\conv	& \convKSF{5}{1}{8}	& relu		& \texttt{72,128,8} 		& \texttt{1008}\\
	\pool	& \poolN				&	/		& \texttt{36,64,12}		& 0	\\	
	\conv	& \convKSF{3}{1}{12}	& relu		& \texttt{36,64,12} 		& \texttt{876}\\
	\pool	& \poolN				&	/		& \texttt{18,32,15} 		& 0	\\
	\conv	& \convKSF{3}{1}{15}	& relu		& \texttt{18,32,15} 		& \texttt{1635}\\
	\pool	& \poolN				&	/		& \texttt{9,16,18}		& 0	\\
	\conv	& \convKSF{3}{1}{18}	& relu		& \texttt{9,16,18} 		& \texttt{2448}\\
	
	\flt		& /					& /			& \texttt{2592}			& \texttt{0}\\
	\dns		& \dnsP{64}			& relu		& \texttt{64}			& \texttt{165952}\\
	\dns		& \dnsP{6}			& softmax	& \texttt{6}				& \texttt{390}\\
	\multicolumn{4}{r}{\textbf{TOTAL}}&{\textbf{172,689}}\\
	\end{tabular}
	%Total params: 172,689
	%Trainable params: 172,689
	%Non-trainable params: 0
\end{table}


\begin{table}[H]
	\centering
	\begin{tabular}{lc}
	\textbf{Param} & \textbf{Value}\\ \hline
	Batch Size 	& 32 \\
	Optimizer 	& Adam \\
	Base lr		& 0.001 \\
	Epochs		& 20 \\
	\end{tabular}
\end{table}


\begin{figure}[H]
	\begin{center}
	\includegraphics[width=\linewidth]{Immagini/conv-pool-1}
	\caption{Graph of the first run}
	\end{center}
\end{figure}
\begin{table}[H]
	\centering
	\begin{tabular}{cccccc}
		\textbf{Run} &\textbf{Loss}&\textbf{V.Loss} &\textbf{Acc.}&\textbf{V.Acc.}&\textbf{$\Delta$ Acc.} \\ \hline
		1   & 1.8558e-04 &  0.0429  & 1.0000    & 0.9933    & 0.0067\\
		2   & 6.4439e-04 &  0.0461  & 1.0000    & 0.9866    & 0.0134\\
		3   & 1.2224e-04 &  0.0133  & 1.0000    & 0.9933    & 0.0067\\
		\textbf{Avg} & \textbf{3.1740e-04} &  \textbf{0.0341}  & \textbf{1.0000}    & \textbf{0.9911}    & \textbf{0.0893}
	\end{tabular}
\end{table}

This model applies the use of the Conv-Pool technique in order to convolve and reduce the data at every step. Since the reduction is performed by various MaxPooling layers, the first two convolutions have now a stride equal to 1. Pooling helps by reducing the tensor dimension while retaining most of the information. 
Moreover, the model has a number of parameters which are an order of magnitude lower than the previous model. Therefore the model has now little to no overfitting and has gained a good amount of validation accuracy.







\newpage
\section{Conclusions}
As seen in the previous sections, after a few experiments, this project has been succesful in classifying the images of Turkish 
Lira banknotes with high accuracy and low data loss.
(altro?) aggiungere migliori risultati ottenuti - 




Esempio di citazione\cite{latexcompanion}

\newpage
\addcontentsline{toc}{section}{References}
\begin{thebibliography}{9}
\bibitem{latexcompanion} 
Michel Goossens, Frank Mittelbach, and Alexander Samarin. 
\textit{The \LaTeX\ Companion}. 
Addison-Wesley, Reading, Massachusetts, 1993.

\bibitem{einstein} 
Albert Einstein. 
\textit{Zur Elektrodynamik bewegter K{\"o}rper}. (German) 
[\textit{On the electrodynamics of moving bodies}]. 
Annalen der Physik, 322(10):891–921, 1905.

\bibitem{knuthwebsite} 
Knuth: Computers and Typesetting,
\\\texttt{http://www-cs-faculty.stanford.edu/\~{}uno/abcde.html}
\end{thebibliography}


\end{document}
