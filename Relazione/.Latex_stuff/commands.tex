\usepackage{float}
\usepackage{url}
\usepackage{xcolor}
\usepackage{pdfpages}
\usepackage{graphicx}
\usepackage{mathtools,amssymb}
\usepackage{amsmath}

%Pro and cons
\usepackage{amssymb}% http://ctan.org/pkg/amssymb
\usepackage{pifont}% http://ctan.org/pkg/pifont
\newcommand{\cmark}{\ding{51}}%
\newcommand{\xmark}{\ding{55}}%
\newcommand{\pro}{\item[\cmark]}
\newcommand{\con}{\item[\xmark]}

\renewcommand{\tilde}{$\sim$}

%sets
\newcommand{\set}[1]{\{#1\}}

% Per numerare i paragrafi
\setcounter{tocdepth}{4}
\setcounter{secnumdepth}{4}
\let\oldpar\paragraph
\renewcommand{\paragraph}[1]{\oldpar{#1}\mbox{}\\}

%Per metter una freccia in modo fast
\newcommand{\fdx}{$\rightarrow$\ }

%Comando per creare nuove definizioni stile blocco
\newcommand{\definition}[2]{
	\begin{table}[H]
	\centering
		\begin{tabular}{|p{0.9\linewidth}}
		\textbf{#1}\\ %Titolo della definzione
		#2\\%Testo della definizione
		\end{tabular}
	\end{table}
	\noindent
}

%Comando esempio
\newcommand{\esempio}[1]{
	\begin{table}[H]
	\centering
		\begin{tabular}{p{0.9\linewidth}}
		\textbf{EX}\\ %Titolo della definzione
		\hline
		#1\\%Testo della definizione
		\end{tabular}
	\end{table}
	\noindent
}

\newcommand{\df}[2]{\definition{#1}{#2}}

\input{insbox.tex}
\newcommand{\lessonDate}[1]{\InsertBoxR{0}{\tiny{#1}}}

\newcommand{\E}{\`E\space}

\usepackage{listings}
\lstset{language=C++,
                keywordstyle=\color{blue},
                stringstyle=\color{red},
                commentstyle=\color{green},
                morecomment=[l][\color{magenta}]{\#}
}
\usepackage{hyperref}
\usepackage{cleveref}

% Tema scuro
%\usepackage{xcolor}
%
%\pagecolor[rgb]{0.25,0.25,0.25}
%\color[rgb]{1,1,1}